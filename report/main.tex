% ai-based-phishing-detection-dissertation/report/main.tex

\documentclass[pdftex,10pt,a4paper,oneside]{article}

% Packages
\usepackage{amsmath}
\numberwithin{equation}{section}
\usepackage{algorithm}
\usepackage{algorithmic}
\usepackage{fancyhdr}
\usepackage{graphicx}
\usepackage{setspace}
\usepackage[]{fncychap}
\usepackage[hyphens]{url}
\usepackage{xcolor}
\usepackage{tabularx}
\usepackage{appendix}
\usepackage[hidelinks]{hyperref}
\usepackage{pdfpages}
\usepackage[round]{natbib}
\usepackage[a4paper,margin=1in]{geometry}
\usepackage{longtable}
\usepackage{caption}
\usepackage{pdflscape}
\usepackage{lscape}

% Fancy header/footer
\fancyhf{}
\pagestyle{fancy}
\renewcommand{\headrulewidth}{0.2pt}
\fancyfoot[C]{\thepage}

% Begin document
\begin{document}

% Preamble
 % Remove headers/footers for the title page
\thispagestyle{empty}

% Double spacing for the title page
\begin{spacing}{2}
    \begin{center}

        % University Crest
        \includegraphics[scale=0.45]{preamble/warwick-crest.pdf}
        \vspace{10mm}

        % Dissertation Title
        \textbf{\LARGE AI Phishing Detection with Explainable AI}
        \vspace{20mm}

        % Student ID
        {\large \textbf{Student ID: 2242090}}
        \vspace{20mm}

        % Supervisor and Department
        {\large Supervisor: Sarah Aktaa}\\
        \textbf{\large Department of Warwick Manufacturing Group (WMG)}\\
        {\large University of Warwick}\\
        {\large Academic Year: 2024-2025}

    \end{center}
\end{spacing}

% Start Roman numbering for front matter
\pagenumbering{roman}

\newpage
% ai-phishing-detection-dissertation/report/preamble/abstract.tex

\section*{Abstract}
\addcontentsline{toc}{section}{Abstract}

Phishing attacks are known to be a big challenge in cybersecurity. Whilst Artificial Intelligence (AI) can help in detecting and preventing such attacks, most models are "black-boxed", and this limits user trust. This study therefore researches the practical development of an AI-powered phishing detection by implementing and evaluating two distinct models: a Random Forest classifier with TF-IDF features and a fine-tuned DistilBERT transformer. Explainable AI (XAI) techniques, such as SHAP for Random Forest and LIME for DistilBERT, were integrated to enhance model transparency. Both models achieved high accuracies of over 98\% on an internal test set comprised of Enron and CEAS 08 corpora. Evaluations on independent, extenal datasets (SpamAssassin, Nigerian Fraud, Nazario) revealed generalisation challenges. Although there was a high precision for phishing instances, there was equally as low recall. The XAI methods integrated provide both global and local explanations, showing how features and words contributed to the model's final outcome.\newline

\large
\noindent This project aligns with the following CyBok Skills: \textbf{Malware \& Attack Technologies}, \textbf{Human Factors}, \textbf{Adversarial Behaviours}.
\newpage
% ai-phishing-detection-dissertation/report/preamble/acknowledgements.tex

\section*{Acknowledgements}
\addcontentsline{toc}{section}{Acknowledgements}

\noindent I would first and foremost like to express my deepest gratitude to my family. To my wonderful parents, \uline{\textbf{Mum}} and \uline{\textbf{Dad}}, thank you for your unwavering support, encouragement, and belief in me throughout my academic journey and during this challenging project. To my sister, \uline{\textbf{Preethika}}, thank you for your constant support and for always being there.\newline

\noindent I would also like to extend my sincere thanks to my friends and peers who provided support, motivation, and a much-needed sense of camaraderie during this demanding period. In no particular order, I would like to thank \uline{\textbf{Goban}}, \uline{\textbf{Caroline}}, \uline{\textbf{Aryan}}, \uline{\textbf{Sarujan}}, \uline{\textbf{Ayesha}}, \uline{\textbf{Thomas}}, \uline{\textbf{Sahaj}}, \uline{\textbf{'Lavan}}, \uline{\textbf{Don}}, \uline{\textbf{Bethel}}, \uline{\textbf{Aman}}, \uline{\textbf{Sara}}, \uline{\textbf{Laura}}, \uline{\textbf{Adina}}, \uline{\textbf{Charlene}}, \uline{\textbf{Louisa}}, \uline{\textbf{Ceaser}}, \uline{\textbf{Obi}}, \uline{\textbf{Joelle}}, \uline{\textbf{Charlie}}, \uline{\textbf{Aqil}}, \uline{\textbf{Ben}}, \uline{\textbf{Karis}}, \uline{\textbf{Harry}}, \uline{\textbf{Thanawitch}}, \uline{\textbf{Margaret}}, \uline{\textbf{Xander}}, and \uline{\textbf{Adam}}. Your discussions, encouragement, and friendship have been truly appreciated.\newline

\noindent Love all of you truly so so much. Couldn't and wouldn't have made it here without any of you in my live. I give me deepest thanks.

\noindent I am also immensely grateful to my supervisor, \uline{\textbf{Sarah}}, for her invaluable guidance, insightful feedback, and consistent support throughout the duration of this dissertation. Her expertise and encouragement were instrumental in navigating the complexities of this research.\newline

\noindent Finally, I acknowledge the developers and maintainers of the open-source software and publicly available datasets that made this research possible.
\newpage
% ai-phishing-detection-dissertation/report/preamble/abbreviations.tex

\section*{Abbreviations}
\addcontentsline{toc}{section}{Abbreviations}

\large
Artificial Intelligence \hfill AI\\
Anti-Phishing Working Group \hfill APWG\\
Bidirectional Encoder Representations from Transformers \hfill BERT\\
Challenge Lab Evaluation Corpus \hfill CEAS\\
Cenetral Processing Unit \hfill CPU\\
Cross Validation \hfill CV\\
Deep Learning \hfill DL\\
Data Version Control \hfill DVC\\
Explainable Boosting Machines \hfill EBM\\
Exploratory Data Analysis \hfill EDA\\
False Negative Rates \hfill FNR\\
General Data Protection Regulation \hfill GDPR\\
Global Digital Population \hfill GDP\\
Gated Recurrent Unit Long Short-Term Memory \hfill GRU-LSTM\\
Graphics Processing Unit \hfill GPU\\
K-Nearest Neighbours \hfill KNN\\
Large File Storage \hfill LFS\\
Large Language Model \hfill LLM\\
Local Interpretable Model-agnostic Explanations \hfill LIME\\
Machine Learning \hfill ML\\
Normalisation Form Compatibly Composition \hfill NKFC\\
Natural Language Processing \hfill NLP\\
Out-of-Vocabulary \hfill OOV\\
Random Forest \hfill RF\\
Recursive Feature Elimination \hfill RFE\\
Area Under the Receiver Operating Characteristic Curve \hfill ROC AUC\\
Scalable Vector Graphics \hfill SVG\\
SHapley Additive exPlanations \hfill SHAP\\
Synthetic Minority Over-sampling Technique \hfill SMOTE\\
Simple Vector Machine \hfill SVM\\
Term Frequency-Inverse Document Frequency \hfill TF-IDF\\
University of New Brunswick \hfill UNB\\
Uniform Resource Locator \hfill URL\\
User Experience \hfill UX\\
Version Control System \hfill VCS\\
Voice over Internet Protocol \hfill VoIP\\
eXplainable Artificial Intelligence \hfill XAI\\
eXplainable Artificial Intelligence with\newline Aquila Optimization Algorithm in Web Phishing Classification \hfill XAIAOA-WPC\\
eXplainable Artificial Intelligence Ensemble-based Filter Feature Selection \hfill XAI-EFFS\\

\newpage
% ai-phishing-detection-dissertation/report/preamble/contents.tex

\setcounter{tocdepth}{2}
\tableofcontents

\newpage
\input{preamble/lists}
\newpage

% Page styling
% ai-phishing-detection-dissertation/report/preamble/chapter-mods.tex

\pagenumbering{arabic} % Arabic page numbering
\lfoot{\centering \thepage} % Centering page numbers in footer


% ai-phishing-detection-dissertation/report/preamble/landscape-style.tex

\fancypagestyle{mylandscape}{
  \fancyhf{} % Clear header and footer
  \fancyfoot[R]{\rotatebox{90}{\centering \thepage}} % Rotate the page number
  \renewcommand{\headrulewidth}{0pt} % No header rule
  \renewcommand{\footrulewidth}{0pt} % No footer rule
}

\newpage

% Sections

% 1: Introduction
\section{Introduction}\label{sec:1-introduction}
% ai-phishing-detection-dissertation/report/sections/1-introduction/background.tex

\subsection{Background}

It is deemed that phishing attacks are one of the most common forms of cyber crimes today, with its targets ranging from individuals to large-scale organisations, with the aim of obtaining sensitive information including personal identification, credentials, and financial data. According to the DataBreach Report by \cite{verizon2023}, it was investigated that social engineering was responsible for over 50\% of all breaches -- a significant point to consider for cybersecurity. The attacks mainly use complex social engineering techniques such as deceptive content, malicious URLs posing as legitimate, and impersonation, all in an attempt to bypass traditional security systems \citep{marett2009effectiveness}. \newline

\noindent Therefore, Artificial intelligence (AI) and machine learning (ML) can serve as effective tools in detecting phishing attempts. Models can be trained on huge datasets on features like suspicious email headers, text anomalies, and malicious links, which humans have a high probability of missing (\citeauthor{chandrasekaran2006phoney}, \citeyear{chandrasekaran2006phoney}; \citeauthor{jain2022survey}, \citeyear{jain2022survey}). However, it is important to note that most traditional AI-phishing detection systems function as a "black-box" model, and they don't offer much transparency into their decision-making processes. Since there is little to no interpretability, it not only reduces trust on the user's side, but the risk that these systems carry inhibit it from being implemented in high-stakes environments such as financial institutions and governmental agencies \citep{ribeiro2016model}. \newline

\noindent Explainable AI (XAI) solve this challenge by attempting to make AI systems more understandable and transparent. Techniques like SHAP (SHapley Additive Explanations) and LIME (Local Interpretable Model-agnostic Explanations) can give insights into the prediction process models undergo (\citeauthor{lundberg2017unified}, \citeyear{lundberg2017unified}; \citeauthor{ribeiro2016model}, \citeyear{ribeiro2016model}). Especially for phishing detection, XAI can very much empower a user's confidence in understanding why an email is flagged, due to either language cues, suspicious links, or email metadata. \newline

\noindent However despite advances in AI and XAI, there still remain gaps into integrating explainability into phishing detection models. Many existing approaches have a prioritisation on accuracy, without much thought for interpretability \citep{hernandes2021phishing}. This means its difficult to strike a balance between performance and transparency. This project therefore addresses to seek this gap, by developing an XAI-enhanced phishing detection system that achieves a high accuracy whilst providing actionable explanations behind its predictions.

\subsection*{Objectives}

\begin{enumerate}
    \item Conduct a comprehensive review of existing AI phishing detection models.
        \begin{itemize}
            \item Review existing AI phishing detection systems.
            \item Identify research gaps to address in relation to interpretability in phishing detection.
            \item Explore XAI techniques, like SHAP and LIME, and their applicability for phishing detection
        \end{itemize}
    \item Identify and implement suitable XAI techniques (e.g., SHAP, LIME).
        \begin{itemize}
            \item Develop upon a traditional model already being used for phishing detection, e.g. Random Forest or Transformer-based models.
            \item Integrate XAI techniques into these models that can explain the model predictions.
            \item Achieve a competetive accuracy that is either on par or better than existing AI-based phishing detection models.
        \end{itemize}
    \item Evaluate the system\textquotesingle s performance in terms of accuracy and interpretability.
        \begin{itemize}
            \item Assess the system on performance metrics such as precision, accuracy, and recall.
            \item Analyse how useful explanations are using standard interpretability metrics.
        \end{itemize}
    \item Compare the usability of the XAI phishing detection model with traditional black-boxed models.
        \begin{itemize}
            \item Conduct small-scale user studies or surveys to determine how effective XAI influences usability and trust -- compared to black-boxed phishing detection models.
            \item Discuss whether or not its worth compromising on accuracy to achieve trust and usability.
        \end{itemize}
\end{enumerate}

% ai-phishing-detection-dissertation/report/sections/1-introduction/research-questions.tex

\subsection{Research questions}

\begin{enumerate}
  \item \textbf{Primary research question}
    \begin{itemize}
      \item \textit{How can Explainable AI (XAI) improve the usability and trustworthiness of AI-based phishing detection systems without compromising on accuracy?}
    \end{itemize}
  \item \textbf{Sub research questions}
    \begin{enumerate}
      \item \uline{\textbf{SUB RESEARCH QUESTION 1}}: \textit{What are the current limitations of AI phishing detection models in terms of their usability and interpretability}?\label{sub-research-q1}
      \item \uline{\textbf{SUB RESEARCH QUESTION 2}}: \textit{How do the interpretability features affect a user's trust building and decision-making processes}?\label{sub-research-q2}
      \item \uline{\textbf{SUB RESEARCH QUESTION 3}}: \textit{How can XAI techniques, i.e. SHAP and LIME, be integrated efficiently on top of existing AI phishing detection models}?\label{sub-research-q3}
      \item \uline{\textbf{SUB RESEARCH QUESTION 4}}: \textit{What trade-offs, if any, arise between the performance and interpretability of the AI phishing detection model}?\label{sub-research-q4}
    \end{enumerate}
\end{enumerate}

\subsection*{Paper structure}

The paper is structured as follows: \hyperref[sec:1-introduction]{Section 1} outlines this study's objectives and research questions. \hyperref[sec:2-literature-review]{Section 2} is the result of a comprehensive literature review into the space of AI-based phishing detection, which neatly progresses onto the relevance of XAI in this field. \hyperref[sec:3-research-methodology]{Section 3} discusses the research methodology, and the practical implementation of the XAI-enhanced phishing detection model. \hyperref[sec:4-results]{Section 4}. \hyperref[sec:5-discussion]{Section 5}. \hyperref[sec:6-conclusion]{Section 6}.


\newpage

% 2: Literature review
\section{Literature review}\label{sec:2-literature-review}
% ai-phishing-detection-dissertation/report/sections/3-research-methodology/model-development/introduction.tex

Taking the preprocessed data that gives a neat, curated feature space, this section will describe the practical design and implementation of an AI phishing detection system that is integrated with XAI explainability methods. Such a setup addresses \hyperref[research-gap-1]{\uline{\textbf{Research Gap 1}}} (performance vs interpretability trade-offs) and \hyperref[research-gap-3]{\uline{\textbf{Research Gap 3}}} (lack of standardised framework for XAI evaluation).\newline

\noindent The model approach took two distinct machine learning models for the task of phishing detection: a traditional ensemble, Random Forest Model, and a transformer-based deep learning model, DistilBERT. The aim is to evaluate and compare their performance with one another, exploring how effective they are in terms of their explainability.\newline

\noindent A methodology like so allows to bridge the gap between traditionally black-boxed AI phishing detection models and more white-box interpretable models, meeting \hyperref[objective-2]{\uline{\textbf{Objective 2}}} whilst simultaneously reaching the defined user-centric goals (\hyperref[objective-4]{\uline{\textbf{Objective 4}}}).


\subsection*{AI-based phishing detection models}
% ai-phishing-detection-dissertation/report/sections/2-literature-review/ai-based-phishing-detection-models/limitations-of-traditional-phishing-detection.tex

\subsubsection*{Limitations of traditional phishing detection}
Whilst traditional phishing detection techniques often heavily rely on blacklists, heuristics, and analysis of content \citep{sheng2009empirical}, they are clearly limited when it comes to more complex attack patterns. It has necessitated more advance phishing detection solutions that are able to adapt to an evolving landscape \citep{andriu2023adaptive}.

% ai-phishing-detection-dissertation/report/sections/2-literature-review/ai-based-phishing-detection-models/ml-models-for-phishing-detection.tex

\subsubsection*{ML models for phishing detection}
AI-driven phishing detection systems have therefore emerged as a potential alternative to traditional machine learning, as stated by \cite{do2022deep}, especially in this context. In particular, they mention how deep learning models, can be applied to detect phishing content in various areas such as websites, emails, mobile devices, VoIP, and so on, achieving competitive accuracies when compared to traditional ML models. \cite{tang2021survey} supports this, by also mentioning how machine learning algorithms, such as neural networks, linear regression, logistic regression, decision trees, SVM, KNN, and random forest, have a high suitablity when it comes to a supervised classification tasks like phishing detection. Specifically, models such as random forest have boasted a high performance in a study conducted by \cite{gupta2021novel}, achieving an accuracy of 99.57\%. Research by \citep{kapoor2024comparative} also advocated for the outstanding performance of random forest, especially in its ability to maintain a classification balance, minimising the risk of false positives and negatives.Other models such as KNN and random forest have achieved similar accuracies, 97.2\%, as seen in the study by \cite{zamir2020phishing}. But all studies note on agree on using a hybrid approach of several models in combination can lead to better detection performance, with an example of using multiple classifiers in \cite{alsariera2020ai} or using a hybrid feature selection process as showcased by \cite{hamid2013using}.

% ai-phishing-detection-dissertation/report/sections/2-literature-review/ai-based-phishing-detection-models/dl-and-transformer-based-models.tex

\subsubsection*{DL and transformer-based models}

% ai-phishing-detection-dissertation/report/sections/2-literature-review/ai-based-phishing-detection-models/practical-benefits-and-future-potential.tex

\subsubsection*{Practical benefits and future potential}


\subsection*{Challenges in AI based phishing detection}
% ai-phishing-detection-dissertation/report/sections/2-literature-review/challenges-in-ai-based-phishing-detection/data-set-and-model-performance-issues.tex

\subsubsection*{Dataset and model performance issues}
Some simpler models, such as decision trees and random forest, are seen to suffer from a case of overfitting due to the imbalance of datasets and high dimensional data, demonstrated in a study by \cite{harikrishnan2018machine}. A large majority of the reasons as to the performance drop-off is due to dataset limitations like lack of specific features or dataset size, as observed by \cite{ahmad2024across}. They mainly noted how datasets struggled to perform well due to a lack of quality and diverse data. Training times were a significant challenge, not just from dataset size, but the inherent nature of deep learning models (consisting of many layers), that might limit their applicability in real-time situations, as discovered by \cite{kapoor2024comparative}, which is also agreed upon by \cite{atlam2022business}.

% ai-phishing-detection-dissertation/report/sections/2-literature-review/challenges-in-ai-based-phishing-detection/adversarial-evolution-and-scalability.tex

\subsubsection*{Adversarial evolution and scalability}
In \cite{kapoor2024comparative}'s research, they also comment on the constant evolution of sophisticated tactics introduced by attackers, such a deep fakes, new social enginnering tactics, and context-aware attacks -- all with the goal of exploiting human technology. They mention that it is important to factor in an "arms race" between AI models being trained on new data and attackers coming up with new intrusion methods. Traditional models, apart from overfitting, suffer from their sensitivity to parameter turning such as SVM \citep{andriu2023adaptive}. Furthermore, it is a challenge to scale these models given the ever-growing needs of an organisation, as the system must be able to both maintain its performance and optimal detection to deal with increasing workloads, with a lack of real-time implementations and studies, observed by \cite{atlam2022business}.

% ai-phishing-detection-dissertation/report/sections/2-literature-review/challenges-in-ai-based-phishing-detection/privacy-and-error-trade-offs.tex

\subsubsection*{Privacy and error trade-offs}

% ai-phishing-detection-dissertation/report/sections/2-literature-review/challenges-in-ai-based-phishing-detection/interpretability-and-xai-challenges.tex

\subsubsection*{Interpretability and XAI challenges}

% ai-phishing-detection-dissertation/report/sections/2-literature-review/challenges-in-ai-based-phishing-detection/ethical-and-human-factor-concerns.tex

\subsubsection*{Ethical and human-factor concerns}


\subsection*{XAI in the context of phishing detection}
% ai-phishing-detection-dissertation/report/sections/2-literature-review/xai-in-the-context-of-phishing-detection/introduction-to-xai-and-its-importance.tex

\subsubsection*{Introduction to XAI and its importance}
There are many studies that put forward the solution of XAI to address interpretability and transparency issues \citep{roshan2022using} associated with systems that employ ML techniques. The goal of XAI here is to serve as a means to understand and inspire confidence in an AI's decision making processes \citep{khanom2025pd_ebm}, from the input all the way through to the output (\citeauthor{jawale2020jeevn}, \citeyear{jawale2020jeevn}; \citeauthor{sanchez2022phishing}, \citeyear{sanchez2022phishing}), with use cases for analysts being able to differentiate between false positives and negatives \citep{van2024applicability}. In particular, there are several existing studies into how XAI plays a role in the context of phishing detection, with a study by \cite{alzahrani2024explainable} proving that both high accuracies (97\%) and reasonable explainability features can be achieved. The importance of such explainable systems is stressed by \cite{shendkar2024enhancing} and further literature comments on how theres been a recent interest for outcome explanations for textual and document based classification tasks (\citeauthor{martens2014explaining}, \citeyear{martens2014explaining}; \citeauthor{lei2016rationalizing}, \citeyear{lei2016rationalizing}).

% ai-phishing-detection-dissertation/report/sections/2-literature-review/xai-in-the-context-of-phishing-detection/user-centric-challenges-and-psychological-factors.tex

\subsubsection*{User-centric challenges and psychological factors}
There also has been studies, such as by \cite{vo2024securing}, that comment on how a lack of understandablility and reasoning in these systems can fail to have humans identify anomalies -- useful for phishing emails where its beneficial for the user to act on the system's warning recommendations without posessing full knowledge of the detection mechanisms. One of the main causes of this hypothetically lies within research proposed by \cite{greco2023explaining}, who claim that one of the reasons that many users often fall victim to phishing attacks is the lack if poorly designed interpretability measures, such as dialog boxes for example, without taking into account a user's psychology. The study points out that most of these UI indicators fail to properly explain the rationale behind their decisions. Supported by a psychological study by \cite{anderson2015polymorphic}, suggests how XAI can give insight into a model's decision making processes to address the issue of the inherent black-boxed nature, where warnings should be of a "polymorphic" nature, i.e. warnings adapt based on the threat which users potentially face, meaning XAI systems can either be oriented to be reveal an AI system's inner workings or be focused to solely provide user understandable explanations (\citeauthor{lipton2018mythos}, \citeyear{lipton2018mythos}; \citeauthor{ribeiro2016model}, \citeyear{ribeiro2016model}).

% ai-phishing-detection-dissertation/report/sections/2-literature-review/xai-in-the-context-of-phishing-detection/xai-techniques-and-methodologies.tex

\subsubsection*{XAI techniques and methodologies}
There are many ways this can be achieved, and this moves the literature review to specific XAI techniques that can be utilised and practically implemented into AI-powered systems, such as SHAP and LIME as a way to offer insights \citep{shendkar2024enhancing} by providing both global and localised explanations, respectively \citep{palaniappan2020malicious}. There is also an additional higher level explainability technique, EBM, referred to by \cite{hernandes2021phishing}, that is a complete white-boxed approach that aims to construct a predictive model that is already inherently explainable by design and does not require additional interpretability tools post processing \citep{greco2023explaining}. Comparing this with LIME, it builds an interpretation based upon existing black-boxed models' outcomes. Both these techniques provide local and global explanations and are model-agnostic, i.e. they can work with any model or classifier that's ML-based \citep{anderson2015polymorphic}. \cite{greco2023explaining} also dives deeper into the different between local and global explanations, adding onto previous understanding. The author mentione that local explanations consist of a "feature importance vector" which is a quantifiable value that shows how much that specific feature contributed to the outcome. This is contrasting to global explanations, as they require both the entire model and its processing mechanisms.

% ai-phishing-detection-dissertation/report/sections/2-literature-review/xai-in-the-context-of-phishing-detection/practical-implementations-and-case-studies.tex

\subsubsection*{Practical implementations and case studies}
A contextual example of these techniques being implemented is the Phishpedia model presented in research by \cite{lin2021phishpedia}, where it takes URLs and returns an explanation to the user based on the legitimacy of its web domain. The important thing here to note is the attention to user feedback, in the form of dialog boxes as previously mentioned. The alert is very visually appealing and informative to the user, with an emphasis on why the URL is potentially malicious, making compairons of known URLs with proper web domains. Another interpretable approach by \cite{bravo2010bridging} is using a phishing email's metadata and content, showing whether each feature (if any), contributed to its final classification of phishing (or not phishing). A very recent study, \cite{lim2025explicate}, designed an XAI and LLM enhanced phishing detection system called EXPLICATE, utilising LIME, SHAP on a DeepSeek v3 base model with a very practical balance achieved between interpretability, accuracy, and efficiency. There are also specialised, innovative XAI approaches, such as XAIAOA-WPC talked about in research by \cite{alotaibi2025explainable} that boasts a high performance rate of 99.29\%, incorporating mainly LIME for its explainability. There are also systems which employ SHAP, such as the integrated intelligence defender framework, CyberDefender proposed by \cite{krishnaveni2024cyberdefender} that utilised a similar custom XAI solution, called XAI-EFFS, which is a filter feature selection that is ensemble-based. This specific solution takes existing hyper-parameters of a GRU-LSTM deep learning model that used optimisation tactics that are Bayesian-inspired. Feature importance analysis can also be supplemented by XAI, as explored by \cite{fajar2024enhancing}, who discovered that the combination of both RFE and XAI could correctly identify distinct dataset features that largely influenced the model's final decision. In this study, the XGBoost and CatBoost models maintained high accuracy and efficiency, with the former being more scalable and the latter being more robust.

% ai-phishing-detection-dissertation/report/sections/2-literature-review/xai-in-the-context-of-phishing-detection/future-directions.tex

\subsubsection*{Future directions}
To conclude, it is safe to say that there are plenty of integrations of XAI with phishing detection models, often achieving competitive accuracies along with interpretability features. A range of XAI techniques are implemented, the most common being SHAP and LIME, with some uses of EBM. However, there lies some areas that need to be addressed as emphasised by most studies' future works.


\newpage

\input{sections/2-literature-review/summary-of-existing-phishing-email-detection-techniques}
\newpage

% ai-phishing-detection-dissertation/report/sections/2-literature-review/summary-of-xai-techniques.tex

\begin{landscape}
  \thispagestyle{mylandscape}
  \subsection*{Summary of XAI techniques}
\end{landscape}

\newpage

\subsection*{Research gaps}

As a result of this literature review, it has underscored several critical gaps in the space of AI-based phishing detection systems -- all which have some direct correlation with this study's objectives and research questions, outlined in \hyperref[sec:1-introduction]{Section 1}.

\begin{enumerate}
  \item \textbf{Interpretability vs. performance trade-offs in phishing detection systems}
  \begin{itemize}
    \item \textbf{Gap}: Higher accuracry models, such as transformers and random forest, often lack many interpretability features. More explainable models like EBM may underperform (\citeauthor{do2024integrated}, \citeyear{do2024integrated}; \citeauthor{greco2023explaining}, \citeyear{greco2023explaining}). There are few studies that find the optimal balance between the two.
    \item \textbf{Aligned with}: Objective 2 (implement XAI techniques), Sub Research Question 4 (trade-offs between performance and interpretability).
  \end{itemize}
  \item \textbf{Lack of a user-centric XAI design}
  \begin{itemize}
    \item \textbf{Gap}: Most of the existing XAI-powered systems focus on technical explanations of outputs, such as with SHAP/LIME. They fail to evaluate how users interact with them (\citeauthor{vo2024securing}, \citeyear{vo2024securing}; \citeauthor{anderson2015polymorphic}, \citeyear{anderson2015polymorphic}).
    \item \textbf{Aligned with}: Objective 4 (usability comparison), Sub Research Question 2 (impact of interpretability on trust).
  \end{itemize}
  \item \textbf{Absence of standardised XAI evaluation metrics}  \begin{itemize}
      \item \textbf{Gap}: There isn't a rigid framework nor a consensus on how to formally assess XAI's effectiveness for phishing detection (\citeauthor{reddy2023explainable}, \citeyear{reddy2023explainable}; \citeauthor{shendkar2024enhancing}, \citeyear{shendkar2024enhancing}).
    \item \textbf{Aligned with}: Objective 3 (evaluate interpretability metrics).
  \end{itemize}
  \item \textbf{Limited real-world validation of XAI phishing detection systems}
  \begin{itemize}
    \item \textbf{Gap}: Most studies test models in controlled environments and often ignore real-world limitations such as privacy (GDPR) compliance and scalability means (\citeauthor{kapoor2024comparative}, \citeyear{kapoor2024comparative}; \citeauthor{atlam2022business}, \citeyear{atlam2022business}).
    \item \textbf{Aligned with}: Sub Research Question 1 (limitations of current models).
  \end{itemize}
  \item \textbf{Inadequate integration of XAI with high performing AI models}  \begin{itemize}
    \item \textbf{Gap}: Whilst its known that SHAP/LIME can be applied to traditional ML models, such as random forest and decision trees, their integration with more complex models like transformers or hybrid architectuers is not substantially explored (\citeauthor{alzahrani2024explainable}, \citeyear{alzahrani2024explainable}; \citeauthor{lim2025explicate}, \citeyear{lim2025explicate}).
    \item \textbf{Aligned with}: Objective 1 (review existing systems), Objective 2 (implement XAI on advaned models).
  \end{itemize}
  \item \textbf{Dynamic adaptability to evolving phishing tactics}
  \begin{itemize}
    \item \textbf{Gap}: Studies have AI models trained on static dataset, and as a result, they fail with novel attack vectors, which includes deepfales and context-aware attacks (\citeauthor{kapoor2024comparative}, \citeyear{kapoor2024comparative}; \citeauthor{atlam2022business}, \citeyear{atlam2022business}). There are few studies that take continous training into consideration, along with XAI to explain the adaptations models make.
    \item \textbf{Aligned with}: Objective 1 (review existing systems), Sub Research Question 1 (limitations of current models).
  \end{itemize}
  \item \textbf{Bias and fairness in XAI interpretability explanations}  \begin{itemize}
      \item \textbf{Gap}: Some studies show that XAI techniques can be influenced from biases in training data, for example flagging emails from specific domains as phishign by default \citep{hanif2021survey}. As of now, there are no studies that account for this bias for phishing-specific XAI outcomes.
    \item \textbf{Aligned with}: Objective 3 (evaluate interpretability metrics).
  \end{itemize}
  \item \textbf{Computational overhead of XAI integration}
  \begin{itemize}
    \item \textbf{Gap}: Studies which attempt real-time deployment are often limited by the comutational costs of the inherent model as well as the additional XAI techniques, i.e. SHAP/LIME latency \citep{kapoor2024comparative}. There isn't much numerical analysis on efficiency and speed.
    \item \textbf{Aligned with}: Research Question 4 (trade-offs between performance and interpretability).
  \end{itemize}
  \item \textbf{Interdisciplinary explanations for non-technical users} 
  \begin{itemize}
    \item \textbf{Gap}: XAI outputs from models that studies have developed often assume technical expertise of the user \citep{greco2023explaining}. There are no current frameworks that adapt XAI explanations to various user roles, such as from a SOC analyst to an end-user.
    \item \textbf{Aligned with}: Objective 4 (usability comparison), Sub Research Question 2 (impact of interpretability on trust). 
  \end{itemize}
  \item \textbf{Privacy respecting XAI for compliance}
  \begin{itemize}
    \item \textbf{Gap}: Since phishing detectors often process sensitive user information extracted from email metadata and content, models have to be privacy respecting. XAI explanations carry the risk of leaking sensitive user data \citep{atlam2022business}. GDPR-compliant XAI implementations are yet to be explored.
    \item \textbf{Aligned with}: Sub Research Question 1 (limitations of current models).
  \end{itemize}
\end{enumerate}


\subsection*{Justification for this study}


\newpage

% 3: Research methodology
\section{Research methodology}\label{sec:3-research-methodology}
% ai-phishing-detection-dissertation/report/sections/3-research-methodology/model-development/introduction.tex

Taking the preprocessed data that gives a neat, curated feature space, this section will describe the practical design and implementation of an AI phishing detection system that is integrated with XAI explainability methods. Such a setup addresses \hyperref[research-gap-1]{\uline{\textbf{Research Gap 1}}} (performance vs interpretability trade-offs) and \hyperref[research-gap-3]{\uline{\textbf{Research Gap 3}}} (lack of standardised framework for XAI evaluation).\newline

\noindent The model approach took two distinct machine learning models for the task of phishing detection: a traditional ensemble, Random Forest Model, and a transformer-based deep learning model, DistilBERT. The aim is to evaluate and compare their performance with one another, exploring how effective they are in terms of their explainability.\newline

\noindent A methodology like so allows to bridge the gap between traditionally black-boxed AI phishing detection models and more white-box interpretable models, meeting \hyperref[objective-2]{\uline{\textbf{Objective 2}}} whilst simultaneously reaching the defined user-centric goals (\hyperref[objective-4]{\uline{\textbf{Objective 4}}}).

\newpage

% 4: Results
\section{Results}\label{sec:4-results}
\newpage

% 5: Discussion
\section{Discussion}\label{sec:5-discussion}
\newpage

% 6: Conclusion
\section{Conclusion}\label{sec:6-conclusion}
\newpage

% 7: References
\bibliographystyle{agsm}
\addcontentsline{toc}{section}{References}
\bibliography{bibliography}
\newpage

% 8: Appendices
% ai-phishing-detection-dissertation/report/appendices/appendices.tex

\section*{Appendices}
\addcontentsline{toc}{section}{Appendices}

\subsection*{Appendix 1: Ethical approval form}
\addcontentsline{toc}{subsection}{Appendix 1: Ethical approval form}
\label{sec:Appendix 1}

\includepdf[pages=-,scale=0.75]{images/ethical-approval-form.pdf}

\newpage

\subsection*{Appendix 2: Ethical courses certificates}
\addcontentsline{toc}{subsection}{Appendix 2: Ethical courses certificates}
\label{sec:Appendix 2}

\begin{figure}[htbp]
    \centering
    \includegraphics[width=0.5\textwidth]{images/ethics-in-research-badge.png}
    \caption{Ethics in research badge}
\end{figure}

\begin{figure}[htbp]
    \centering
    \includegraphics[width=0.5\textwidth]{images/ethical-approval-process-badge.png}
    \caption{Ethical approval badge}
\end{figure}

\begin{figure}[htbp]
    \centering
    \includegraphics[width=1\textwidth]{images/epigeum-certificate.png}
    \caption{Epigeum certificate}
\end{figure}

\begin{figure}[htbp]
    \centering
    \includegraphics[width=1\textwidth]{images/information-security-smart-certificate.png}
    \caption{Information SMART certificate}
\end{figure}

\newpage

\subsection*{Appendix 3: Project plan}
\addcontentsline{toc}{subsection}{Appendix 3: Project plan}
\label{sec:Appendix 3}

\begin{longtable}{|p{5cm}|p{6cm}|p{4cm}|}
    \hline
    \textbf{Phase} & \textbf{Tasks} & \textbf{Duration} \\
    \hline
    \endfirsthead
    \hline
    \textbf{Phase} & \textbf{Tasks} & \textbf{Duration} \\
    \hline
    \endhead
    
    \hline
    \endfoot
    
    \hline
    \endlastfoot
    
    Research and Planning & 
    \begin{itemize}
        \item Conduct literature review.
        \item Refine research questions and objectives.
        \item Identify datasets (e.g., PhishTank, Kaggle).
    \end{itemize} & 
    End Jan – Early Feb (1–2 weeks) \\
    \hline
    
    Dataset Preparation & 
    \begin{itemize}
        \item Clean and preprocess data.
        \item Perform feature extraction (e.g., URLs, metadata).
        \item Split data into training, validation, and test sets.
    \end{itemize} & 
    Early Feb (1–2 weeks) \\
    \hline
    
    Model Development & 
    \begin{itemize}
        \item Develop baseline model (e.g., Random Forest, Logistic Regression).
        \item Experiment with advanced models (e.g., BERT).
        \item Train and validate models.
    \end{itemize} & 
    Mid-Feb – Early Mar (3 weeks) \\
    \hline
    
    Integration of XAI & 
    \begin{itemize}
        \item Implement SHAP and LIME for interpretability.
        \item Visualize explanations (e.g., feature importance, email components).
        \item Ensure explanation clarity and usability.
    \end{itemize} & 
    Early Mar – Mid-Mar (2 weeks) \\
    \hline
    
    Evaluation and Comparison & 
    \begin{itemize}
        \item Evaluate model performance (accuracy, F1-score).
        \item Assess interpretability using metrics (faithfulness, stability).
        \item Compare XAI-enhanced system to black-box models.
    \end{itemize} & 
    Mid-Mar – End Mar (2 weeks) \\
    \hline
    
    Report Writing & 
    \begin{itemize}
        \item Write methodology, results, and discussion sections.
        \item Proofread and finalize dissertation.
        \item Integrate appendices (e.g., Gantt chart, ethical approval).
    \end{itemize} & 
    April – Mid-May (5–6 weeks) \\
    \hline
    
    \end{longtable}
    


\end{document}
