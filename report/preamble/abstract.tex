% ai-phishing-detection-dissertation/report/preamble/abstract.tex

\section*{Abstract}
\addcontentsline{toc}{section}{Abstract}

Phishing attacks are known to be a big challenge in cybersecurity. Whilst Artificial Intelligence (AI) can help in detecting and preventing such attacks, most models are "black-boxed", and this limits user trust. This study therefore researches the practical development of an AI-powered phishing detection by implementing and evaluating two distinct models: a Random Forest classifier with TF-IDF features and a fine-tuned DistilBERT transformer. Explainable AI (XAI) techniques, such as SHAP for Random Forest and LIME for DistilBERT, were integrated to enhance model transparency. Both models achieved high accuracies of over 98\% on an internal test set comprised of Enron and CEAS 08 corpora. Evaluations on independent, extenal datasets (SpamAssassin, Nigerian Fraud, Nazario) revealed generalisation challenges. Although there was a high precision for phishing instances, there was equally as low recall. The XAI methods integrated provide both global and local explanations, showing how features and words contributed to the model's final outcome.\newline

\large
\noindent This project aligns with the following CyBok Skills: \textbf{Malware \& Attack Technologies}, \textbf{Human Factors}, \textbf{Adversarial Behaviours}.