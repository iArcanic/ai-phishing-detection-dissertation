% ai-phishing-detection-dissertation/report/section/5-discussion/introduction.tex

Taking the empirical results from the previous section concerning the two phishing detection models -- Random Forest and DistilBERT -- along with the visual diagrams of the SHAP and LIME explanations will form an analysis for a more comprehensive and detailed understanding, linking the broader scope and context.\newline

\noindent This discussion's main aim is to observe how the model performance achieved the results recorded, giving reasons to the success on the internal test data and the short comings for the generalisation capabilities on independent, external datasets. In addition, the utility of the XAI explanations from SHAP and LIME techniques for the Random Forest and DistilBERT models respectively, will be examined in terms of their clarity and how well they are understood.\newline

\noindent Additionally, the core research questions as well as the objectives defined at the beginning of the study in \hyperref[sec:1-introduction]{\uline{\textbf{Section 1}}} will be evaluated to see how well they are met given the achieved results and subsequent analysis. The discussion takes into the account the inherent advantages and disadvantages of each modeling approach. And finally, limitations of the study will be addressed for future works mentioned in the next section.
