% ai-phishing-detection-dissertation/sections/6-conclusion/future-works.tex

\subsection{Future works}
The findings and limitations of this study bring up many areas for future work:

\begin{itemize}
  \item \textbf{Improving model generalisation}: There should be more work focused on improving the models' performance on general phishing emails, and build resilience to changes in semantic contexts and domain shifts. This might be training with a larger, more diverse dataset.
  \item \textbf{Comprehensive XAI evaluation and user studies}: Conduct more standardised, rigorous user feedback evaluations on different aspects of XAI explanations with a wide range and large number of participants. This is crucial since it tests if the model is ready for deployment in real-world settings.
  \item \textbf{Adversarial attack robustness}: Models should also be made security aware and have some mechanisms preventing the bypass of filters via carefully crafted attacks.
  \item \textbf{Real-world deployment considerations}: Consider how such a theoretical model fares in real-world settings, such as complex security workflows to various, complex human interactions.
\end{itemize}

\noindent In conclusion, there is a lot of potential for combining traditional AI-powered phishing detection systems with the novelty and utility of various XAI techniques for a more transparent process. It is clear however, that continuous research and development is required in this area, concerning the data used, adaptability of models, and how user-centric these systems are. These are all in an effort to make the fight against phishing attacks much easier.
