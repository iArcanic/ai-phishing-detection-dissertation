% ai-phishing-detection-dissertation/sections/6-conclusion/conclusion.tex

To conclude, this dissertation had the primary goal of developing an AI-powered phishing detection system enhanced with Explainable AI (XAI) techniques. This is in the aim to improve trust in these systems, and their usability, as well as their practicality in real-world settings. To achieve this, as part of this study, two machine learning models were chosen. A Random Forest classifier trained upon TF-IDF features, and a fine-tuned DistilBERT transformer-based model. The XAI techniques, SHAP and LIME, were integrated respectively as a way to reveal some insight into the classification process.\newline

\noindent The research shows that these models are able to achieve such a high performance (greater than 98-99\%) on an internal test set comprising of Enron and CEAS 2008 corpora, meaning the models are capable of identifying data of the same distribution. The integration of the XAI techniques, SHAP and LIME, serve to provide both local and global explanations, highlighting specific features and textual components (words or phrases) that contributed most to the model's final prediction. This is a way to offer transparency into a model's inner-workings. Conducted user feedback largely confirmed that these explanations were helpful in explaining the classification process, but as well as increasing the trust a user has in AI-powered systems.\newline

\noindent It's important to note that the main challenge that was found in the study is both the model's ability to generalise on independent external datasets (SpamAssassin, Nazaio, Nigerian Fraud). While it was observed that precision remained high for the phishing class, recall dropped significantly, and a lot of phishing emails were missed in these datasets. It's suspected this is due to a domain shift and change in vocabulary that these sets inherently boasted, that were not covered in the Enron+CEAS 2008 corpus that the models were trained upon.\newline

\noindent To answer the primary research question, this study proposes that XAI can very much be integrated into an AI phishing detection system without comprising performance at all on a similar training distribution. The generated explanations are a guaranteed path therefore to improve trust and usability. But it's currently held back by the model's lack of generalisation capabilities. In a real-world scenario this is not feasible since a robust performance across a wide range of phishing attacks is required.\newline

\noindent This study has several key limitations, including the diversity of the initial training data, and the limited scope of XAI integration and subsequent evaluations.
