% ai-phishing-detection-dissertation/report/sections/2-literature-review/challenges-in-ai-based-phishing-detection/privacy-and-error-trade-offs.tex

\subsubsection*{Privacy and error trade-offs}
GDPR comes into play here as these systems use user information from email content and metadata, so they need to be ensure compliance and have the appropriate privacy mechanisms implemented. Models are also likely to fall victim of flagging false positives and negatives, and this is a vital aspect in keeping them from being deployed in a real-time setting, emphasised by \cite{vishwanath2011people}. False positives can logically cause disruption and inconvenience for businesses and users respectively, where as false negatives are phishing emails that may bypass filters introducing a vulnerability. It is vital for models to prioritise a balance between these two categories of errors for a practical phishing detection system.
