% ai-phishing-detection-dissertation/report/sections/2-literature-review/challenges-in-ai-based-phishing-detection/data-set-and-model-performance-issues.tex

\subsubsection*{Dataset and model performance issues}
Some simpler models, such as decision trees and random forest, are seen to suffer from a case of overfitting due to the imbalance of datasets and high dimensional data, demonstrated in a study by \cite{harikrishnan2018machine}. A large majority of the reasons as to the performance drop-off is due to dataset limitations like lack of specific features or dataset size, as observed by \cite{ahmad2024across}. They mainly noted how datasets struggled to perform well due to a lack of quality and diverse data. Training times were a significant challenge, not just from dataset size, but the inherent nature of deep learning models (consisting of many layers), that might limit their applicability in real-time situations, as discovered by \cite{kapoor2024comparative}, which is also agreed upon by \cite{atlam2022business}.
