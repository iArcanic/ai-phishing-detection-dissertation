% ai-phishing-detection-dissertation/report/sections/2-literature-review/justification-for-this-study/theoretical-and-practical-contributions.tex

\subsubsection*{Theoretical and practical contributions}
In summary, this work offers the following:

\begin{itemize}
  \item \textbf{Theoretical novelty}: A framework for XAI-powered phishing detection model evaluation, addressing \hyperref[research-gap-1]{\uline{\textbf{Research Gap 1}}}, \hyperref[research-gap-2]{\uline{\textbf{Research Gap 2}}}, and \hyperref[research-gap-3]{\uline{\textbf{Research Gap 3}}}.
  \item \textbf{Practical impact}: A user-tested design, with principles for interpretable phishing alerts, meeting \hyperref[research-gap-2]{\uline{\textbf{Research Gap 2}}} and \hyperref[research-gap-9]{\uline{\textbf{Research Gap 9}}}.
  \item \textbf{Methodological rigor}: Standardised metrics and guidelines for reproducibility, satisfying \hyperref[research-gap-3]{\uline{\textbf{Research Gap 3}}}.
\end{itemize}

\noindent The targeted research gaps ensure that he study not only answers its research questions outlined, but served to provide actionable details for industrial deployment. This matches requirements for a deployable yet trustworthy AI-based phishing detection system as called for by \cite{atlam2022business} and \cite{lim2025explicate}.
