% ai-phishing-detection-dissertation/report/sections/2-literature-review/ai-based-phishing-detection-models.tex

\subsection*{AI-based phishing detection models}

Whilst traditional phishing detection techniques often heavily rely on blacklists, heurestics, and analysis of content \citep{sheng2009empirical}, they are clearly limited when it comes to more complex attack patterns. It has necessitated more advance phishing detection solutions that are able to adapt to an evolving landscape \citep{andriu2023adaptive}. AI-driven phishing detection systems have therefore emerged as a potential alternative to traditional machine learning, as stated by \cite{do2022deep}, especially in this context. In particular, they mention how deep learning models, can be applied to detect phishing content in various areas such as websites, emails, mobile devices, VoIP, and so on, achieving competitive accuracies when compared to traditional ML models. \cite{tang2021survey} supports this, by also mentioning how machine learning algorithms, such as neural networks, linear regression, logistic regression, decision trees, SVM, KNN, and random forest, have a high suitablity when it comes to a supervised classification tasks like phishing detection. Specifically, models such as random forest have boasted a high performance in a study conducted by \cite{gupta2021novel}, achieving an accuracy of 99.57\%. Research by \citep{kapoor2024comparative} also advocated for the outstanding performance of random forest, especially in its ability to maintain a classification balance, minimising the risk of false positives and negatives.Other models such as KNN and random forest have achieved similar accuracies, 97.2\%, as seen in the study by \cite{zamir2020phishing}. But all studies note on agree on using a hybrid approach of several models in combination can lead to better detection performance, with an example of using multiple classifiers in \cite{alsariera2020ai} or using a hybrid feature selection process as showcased by \cite{hamid2013using}. Additionally, transformer based models have been shown to perform equally as well, with a study by \cite{do2024integrated} presenting a model with a 99.71\% detection accuracy. Another study by \cite{shirazi2022towards} agrees with this notion, particularly highlighting the significance of pre-trained transformer models showing substantial performance compared to other current approaches, despite their lower accuracy scores. Their advantage lies in the fact that they do not require pre-processing as certain features -- making them highly adaptable for feature-driven detection. Not ony that, but the study claims that transformer-based models are more practical for organisations, since they are faster and require less training time \citep{shirazi2022towards}. All in all, it is safe to conclude that AI-driven phishing email detection serves as the logical choice to be integrated (or replace) existing solutions. Due to the easy reproducability, extensibility, and adaptable nature of these systems, present promise in this field \citep{bauskar2024ai}.

