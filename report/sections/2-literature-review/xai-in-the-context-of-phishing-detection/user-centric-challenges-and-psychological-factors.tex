% ai-phishing-detection-dissertation/report/sections/2-literature-review/xai-in-the-context-of-phishing-detection/user-centric-challenges-and-psychological-factors.tex

\subsubsection*{User-centric challenges and psychological factors}
There also has been studies, such as by \cite{vo2024securing}, that comment on how a lack of interpretability and reasoning in these systems can fail to have humans identify anomalies -- useful for phishing emails where its beneficial for the user to act on the system's warning recommendations without possessing full knowledge of the detection mechanisms. One of the main causes of this hypothetically lies within research proposed by \cite{greco2023explaining}, who claim that one of the reasons that many users often fall victim to phishing attacks is the lack if poorly designed interpretability measures, such as dialog boxes for example, without taking into account a user's psychology. The study points out that most of these UI indicators fail to properly explain the rationale behind their decisions. Supported by a psychological study by \cite{anderson2015polymorphic}, suggests how XAI can give insight into a model's decision making processes to address the issue of the inherent black-boxed nature, where warnings should be of a "polymorphic" nature, i.e. warnings adapt based on the threat which users potentially face, meaning XAI systems can either be oriented to be reveal an AI system's inner workings or be focused to solely provide user understandable explanations (\citeauthor{lipton2018mythos}, \citeyear{lipton2018mythos}; \citeauthor{ribeiro2016model}, \citeyear{ribeiro2016model}).
