% ai-phishing-detection-dissertation/report/sections/2-literature-review/xai-in-the-context-of-phishing-detection/practical-implementations-and-case-studies.tex

\subsubsection*{Practical implementations and case studies}
A contextual example of these techniques being implemented is the Phishpedia model presented in research by \cite{lin2021phishpedia}, where it takes URLs and returns an explanation to the user based on the legitimacy of its web domain. The important thing here to note is the attention to user feedback, in the form of dialog boxes as previously mentioned. The alert is very visually appealing and informative to the user, with an emphasis on why the URL is potentially malicious, making compairons of known URLs with proper web domains. Another interpretable approach by \cite{bravo2010bridging} is using a phishing email's metadata and content, showing whether each feature (if any), contributed to its final classification of phishing (or not phishing). A very recent study, \cite{lim2025explicate}, designed an XAI and LLM enhanced phishing detection system called EXPLICATE, utilising LIME, SHAP on a DeepSeek v3 base model with a very practical balance achieved between interpretability, accuracy, and efficiency. There are also specialised, innovative XAI approaches, such as XAIAOA-WPC talked about in research by \cite{alotaibi2025explainable} that boasts a high performance rate of 99.29\%, incorporating mainly LIME for its explainability. There are also systems which employ SHAP, such as the integrated intelligence defender framework, CyberDefender proposed by \cite{krishnaveni2024cyberdefender} that utilised a similar custom XAI solution, called XAI-EFFS, which is a filter feature selection that is ensemble-based. This specific solution takes existing hyper-parameters of a GRU-LSTM deep learning model that used optimisation tactics that are Bayesian-inspired. Feature importance analysis can also be supplemented by XAI, as explored by \cite{fajar2024enhancing}, who discovered that the combination of both RFE and XAI could correctly identify distinct dataset features that largely influenced the model's final decision. In this study, the XGBoost and CatBoost models maintained high accuracy and efficiency, with the former being more scalable and the latter being more robust.
