% ai-phishing-detection-dissertation/report/sections/2-literature-review/research-gaps/research-gap-6.tex

\subsubsection*{Research gap 6: Dynamic adaptability to evolving phishing tactics}\label{research-gap-6}
Studies have AI models trained on static dataset, and as a result, they fail with novel attack vectors, which includes deepfakes and context-aware attacks (\citeauthor{kapoor2024comparative}, \citeyear{kapoor2024comparative}; \citeauthor{atlam2022business}, \citeyear{atlam2022business}). There are few studies that take continous training into consideration, along with XAI to explain the adaptations models make. Aligned with \hyperref[objective-1]{\uline{\textbf{Objective 1}}} and \hyperref[sub-research-q1]{\uline{\textbf{Sub Research Question 1}}}.
