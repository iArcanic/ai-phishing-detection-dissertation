% ai-phishing-detection-dissertation/report/sections/3-research-methodology/data-preprocessing/introduction.tex

Phishing detection, one based on a machine learning model, requires the transformation of raw, noise data, into a more structured format. This is to effectively capture attack signatures from datasets well, with the goal of ignoring irrelevant information. This section is responsible for detailing a domain-specific preprocessing pipeline for the selected datasets. This pipeline aims to:

\begin{itemize}
  \item \textbf{Preserve threat indicators}: Cleaning but strategically keep security-critical elements, as identified in \cite{greco2023explaining} and \cite{lin2021phishpedia}.
  \item \textbf{Optimise for explainability}: Feature engineering has a priority of human-interpretable aspects which are compatible with XAI techniques (e.g. SHAP/LIME) \citep{shendkar2024enhancing}.
  \item \textbf{Address phishing-specific challenges}: Issues such as imbalanced class distributions, adversarial attack tactics, and multimodal data, are taken care of with inspiration from methodologies in works such as by \cite{zamir2020phishing} and \cite{ahmad2024across}.
\end{itemize}

\noindent The pipeline, described in this section, consists of the following stages:

\begin{itemize}
  \item \textbf{Cleaning \& normalisation}: Sanitise emails and URLs with respect to a security context.
  \item \textbf{Text-specific processing}: Utilise NLP-like techniques that are specifically optimised for phishing linguistics.
  \item \textbf{Feature engineering}: Extract features which are interpretable, for subsequent XAI integration.
  \item \textbf{Balancing \& splitting}: Combat the effects of inherent class imbalance whilst aiming to prevent data leakage.
\end{itemize}

\noindent Such an approach ensures reproducibility and meets \hyperref[research-gap-5]{\uline{\textbf{Research Gap 2}}} (interpretability) and \hyperref[research-gap-5]{\uline{\textbf{Research Gap 5}}} (generalisability) through preserving a loyalty to textual semantics when performing the various data operations.
