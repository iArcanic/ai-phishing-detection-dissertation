% ai-phishing-detection-dissertation/report/sections/3-research-methodology/data-preprocessing/dataset-balancing-and-splitting.tex

\subsubsection*{Dataset balancing and splitting}
As mentioned previously, the performance of an ML algortithm largely depends upon the quality of the data its trained upon. Unfortnately, a lot of datasets unintentionally exhibit sever class imbalance, often in less than 5\% of phishing samples. To combat this, this study uses these following strategies:

\begin{itemize}
  \item \textbf{Synthetic Minority Oversampling (SMOTE)}:
  \begin{itemize}
    \item Synthetic phishing samples can be generated in the feature space with k-NN, where $k = 5$. 
    \item This should only be applied to training data to prevent leakages \citep{ahmad2024across}.
    \item Minority-class patterns, which are critical and under-represented, are preserved with SMOTE.
  \end{itemize}
  \item \textbf{Strategic undersampling}:
  \begin{itemize}
    \item Majority classes should be clustered (legitimate emails) with k-means, where here, $k = 10$. Each cluster should be sampled proportionally \citep{zamir2020phishing}.
    \item Allows for classes to retain their diversity, especially for legitimate traffic.
  \end{itemize}
  \item \textbf{Cost-sensitive training}:
  \begin{itemize}
    \item Class weights should be assigned, that are inversely proportional to frequency (e.g., phishing: 0.9, legitimate: 0.1) during model training \citep{gupta2021novel}.
  \end{itemize}
\end{itemize}

\noindent Furthermore, the data will follow a three-tier partitioning set up to allow for a rigorous evaluation process whilst simulating real-world conditions:

\begin{itemize}
  \item \textbf{Time-aware splitting}
  \begin{itemize}
    \item Emails should be sorted by timestamps -- if available.
    \item Training with 70\% of the oldsest data, validation on 15\%, and test on 15\% on the newest \citep{kapoor2024comparative}.
    \item Can test a model on new and emerging attack methods.
  \end{itemize}
  \item \textbf{Stratified splitting (fallback)}:
  \begin{itemize}
    \item If timestamps are not available, then a fallback option would be to use stratified sampling to retrain class ratios in all splits.
    \item Use 10-fold cross-validation for smaller datasets, such as Nigerian Fraud.
  \end{itemize}
  \item \textbf{Adversarial hold out set}:
  \begin{itemize}
    \item Reserve 5\% of phishing samples from the test set to represent "zero-day" attacks, which are never seen during taining nor validation \citep{atlam2022business}.
  \end{itemize}
\end{itemize}

\noindent Here are some points that concern the practical implementation of this:

\begin{itemize}
  \item \textbf{Tooling}: Use "\texttt{imbalanced-learn}" for SMOTE and "\texttt{sckit-learn}" for clustering and/or splitting.
  \item \textbf{Reproducibility}: Use fixed random seeds (42) for stochastic operations.
  \item \textbf{Edge cases}: Distribute Nigerian Fraud samples across all splits to test the models responses for social engineering.
\end{itemize}
