% ai-phishing-detection-dissertation/report/sections/3-research-methodology/data-preprocessing/text-specific-processing.tex

\subsubsection*{Text-specific processing}
Phishing texts require adapted NLP processing with the goal of reducing noise as well as preserving their original malicious intent. Here, techniques from studies by \cite{do2022deep} and \cite{shirazi2022towards} will be utilised here, especially for normalisation and tokenisation that are both security aware.

\begin{enumerate}
  \item \textbf{Security-conscious tokenisation}
  \begin{itemize}
    \item \textit{Email/URL tokenisation}:
    \begin{itemize}
      \item SpaCy's English pipeline can have custom rules to preserve critical security n-grams, such as "verify account" or "password reset", as singular tokens \citep{van2024applicability}.
      \item Concatenated words should be split, for e.g. "PayPalCustomerService" $\Rightarrow$ ["PayPal", "Customer", "Service"].
      \item With URLs, tokenisation should be Regex-based, especially concerning domains and paths, for e.g. "bank[.]com/login" $\Rightarrow$ ["bank", "com", "login"] \citep{palaniappan2020malicious}.
    \end{itemize}
    \item \textit{Handling special cases}:
    \begin{itemize}
      \item Original casing, for branded terms like "AppleID" vs "appleid", should be preserved.
      \item Suspisious unicode blogs, should be isolated and flagged, for e.g. Cyrillic in English emails \citep{andriu2023adaptive}.
    \end{itemize}
  \end{itemize}
  \item \textbf{Stopword processing}
  \begin{itemize}
    \item \textit{Context-aware stopword removal}:
    \begin{itemize}
      \item Not all text needs to be tokenised, so a custom st word list can be applied to exclude security verbs (e.g. "verify" or "authenticate") and urgency markers (e.g. "urgent" or "immediately").
      \item Terms of negation, like "not" or "never", should be maintained as they are vital for scam email wording.
    \end{itemize}
    \item \textit{Domain-specific filtering}:
    \begin{itemize}
      \item In the context of financial phishing, monetary terms like "USD" or "wire transfer", should be kept.
      \item In the context of credential theft, authentication phrases, i.e. "log in" or "credentials", should be preserved \citep{bravo2010bridging}.
    \end{itemize}
  \end{itemize}
  \item \textbf{Stemming vs lemmatisation}
  \begin{itemize}
    \item \textit{Lemmatisation preferred}:
    \begin{itemize}
      \item WordNet-based lemmatisation can maintain meaning of words (e.g. "phishing" $\Rightarrow$ "phish" and not "phish-ing").
      \item Also keep verb tense in threats ("Your account will be sus[emded") \citep{martens2014explaining}.
    \end{itemize}
    \item \textit{Exceptions}:
    \begin{itemize}
      \item Proper nouns like "Microsoft" or "BankofAmerica" should not be lemmatised.
      \item URL fragments should remain exactly as they are, e.g. "/login/php".
    \end{itemize}
  \end{itemize}
\end{enumerate}
