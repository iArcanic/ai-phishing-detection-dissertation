% ai-phishing-detection-dissertation/report/sections/3-research-methodology/model-development/security-and-privacy-considerations.tex

\subsubsection{Security and privacy considerations}
Although this project's broader scope encompasses techniques that preserve security and user privacy, it is important to note that this study's implementation phase had a core focus on the development and explainability of the detection models. To address this in some respect, basic input sanitisation was applied, such as the removal and defanging of URLs and email addresses from textual components used by the Random Forest model. General text cleaning for both models was performed, and this includes NKFC normalisation, tokenisation, and lemmatisation, during data preprocessing. However, the models were not designed to be deployed in a production environment, and therefore, the security and privacy aspects of the models were not a primary focus. The models are intended for research purposes only, and any deployment in a real-world scenario would require additional security measures to ensure user privacy and data protection.\newline

\noindent More advanced techniques such as adversarial attack training, dynamic risk adaptations, and privacy-respecting XAI outputs (redaction of sensitive user data from explanations) are beyond the scope of the project's current implementation due to time and resource limitations. However, these techniques are important considerations for future work and should be integrated into the models to enhance security and privacy features. It's vital for the models to be evaluated in production-like environments, to ensure that they can withstand real-world attacks and maintain user privacy.
