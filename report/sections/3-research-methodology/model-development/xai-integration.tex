% ai-phishing-detection-dissertation/report/sections/3-research-methodology/model-development/xai-integration.tex

\subsubsection*{XAI integration}
The core theme of this project is XAI, providing transparency and insight into the decision-making processes of the two trained models. These specific XAI techniques were integrated.

\begin{itemize}
  \item \textbf{Random forest (SHAP)}:
  \begin{itemize}
    \item SHAP was used for this model to explain its predictions using "\texttt{shap.Explainer}", initialised with the trained Random Forest model the the TF-IDF features. This allows for:
    \begin{itemize}
      \item \textit{Global explanations}: SHAP summary plots were generated to serve as a visualisation of the overall TF-IDF feature importance (terms/n-grams) across the dataset.
      \item \textit{Local explanations}: Waterfall plots were generated for individual email predictions, with information on how certain features may (or may not) have contributed to a prediction.
    \end{itemize}
  \end{itemize}
  \item \textbf{DistilBERT (LIME)}:
  \begin{itemize}
    \item LIME for the deep learning model was used for local, word-level explanations with the "\texttt{LimeTextExplainer}" function from the "\texttt{lime}" library.
    \item A predictor function was required to allow LIME to get probability predictions from the DistilBERT model for text samples.
    \item Words were highlighted via these LIME explanations from the input text that contributed most to the final model's classification for each individual email.
  \end{itemize}
\end{itemize}
