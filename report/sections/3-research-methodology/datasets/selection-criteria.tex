% ai-phishing-detection-dissertation/report/sections/3-research-methodology/dataset=selection-and-evaluation/selection-criteria.tex

\subsubsection*{Selection criteria} 
To balance both the the relevancy and feability to this study, the following datasets were chosen based on:

\begin{itemize}
  \item \textbf{Relevancy to phishing}: Datasets with a priority to labels that were suited to phishing, such as PhishTank and Nazario Spam.
  \item \textbf{Size}: There should be at least a minimum of 5,000 samples, disregarding niche datasets like the Nigerian Fraud Dataset.
  \item \textbf{Updated}: Prefering datasets that are post-2010 where possible, minus the Enron Phishing Email Dataset for NLP baselines.
  \item \textbf{Diversity of features}: Datasets can be combined, such as URL features in PhishTank and emails in Enron/CEAS, for a diverse, feature-rich, overall dataset.
\end{itemize}

