% ai-phishing-detection-dissertation/report/sections/3-research-methodology/datasets/sources.tex

\subsubsection*{Sources}
The study has a priority for datasets that cover a wide range of phishing social engineering tactics. This is as an assurance for diversity and relevance. Key candidates here include:

\begin{itemize}
  \item \textbf{Email-based phishing}:
  \begin{itemize}
    \item \textit{Enron Phishing Email Dataset}: Consists of over 500,000 corporate emails (but can vary depending upon filters applied), with a mix of both ham and phishing. \citep{klimt2004enron}.
    \item \textit{CEAS 2008 Challenge Corpus}: A compilation of over 25,000 spam/phishing emails via a competition, with contributions from the public \citep{cormack2008email}.
    \item \textit{SpamAssassin Public Corpus}: A collection of ham, spam, and phishing subsets maintained by the Apache Software Foundation \citep{spamassassin2003}.
    \item \textit{Nazario Spam Dataset}: Consists of around 4,000 emails or from 2004-2007 from sources such as deliberate honeypots or public archives \citep{nazario2007phishing}.
  \end{itemize}
\item \textbf{URL-based phishing}:
\begin{itemize}
  \item \textit{PhishTank}: A database of 10,000 live phishing URLs that are updated constantly \citep{phishTank2023}.
  \item \textit{UNB Phishing Dataset}: A collection of 40,000 phishing email URLs with diverse features \citep{unb2016phishing}.
\end{itemize}
\item \textbf{Social engineering/specialised}:
  \begin{itemize}
    \item \textit{Nigerian Fraud Dataset}: A publically available and detailed dataset compiled of 1,000 "419 scam" emails \citep{champa2024phishing}.
    \item \textit{Ling-Spam Corpus}: A collection of 2,400 public phishing emails from the "Linguist List" online forum \citep{ling2005spam}.
  \end{itemize}
\end{itemize}

