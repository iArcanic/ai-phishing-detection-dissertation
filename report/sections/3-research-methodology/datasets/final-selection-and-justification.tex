% ai-phishing-detection-dissertation/report/sections/3-research-methodology/datasets/final-selection-and-justification.tex

\subsubsection*{Final selection and justification}
From the evaluation above, the selected datasets that were chosen for this research to address the varied nature of phishing attackers are detailed below. A breakdown is provided, that includes their synergies, and close alignment with the identified research gaps.

\begin{itemize}
  \item \textbf{Primary training dataset: Enron}
  \begin{itemize}
    \item Served as the primary source of legitimate ("ham", label 0) emails, and was split into training, validation, and internal sets.
    \item Chosen for its large value of real-world corporate emails, which are relatively preprocessed already and support a format for generating a "ham" corpus.
  \end{itemize}
  \item \textbf{Secondary training dataset: CEAS 2008}
  \begin{itemize}
    \item Served as a primary source for phishing/spam emails, i.e. label 1, and was split into training, validation, and internal sets.
    \item Has clearly identifiable labels and columns compared to its raw TREC format, and it's also a known corpus for spam/phishing, providing plenty of relevant examples for malicious emails.
  \end{itemize}
  \item \textbf{Primary independent testing dataset: SpamAssassin}:
  \begin{itemize}
    \item The spam components served as an independent test set for evaluating model performance on general spam/phishing emails (with label 1), which are unseen during training.
    \item The ham components had the role of testing the models' ability to correctly classify legitimate emails (label 0), with a "hard" and "easy" set.
    \item Chosen due to its wide recognition in its benchmarking for spam filtering, with distinct separation between ham and spam categories, as well as a further division in terms of difficulty.
    \item Robust and well-labelled data, suitable for testing.
  \end{itemize}
  \item \textbf{Secondary independent testing dataset: Nigerian Fraud}
  \begin{itemize}
    \item Another specialised independent test set to to evaluate the model's performance on a specific type of phishing email tactic, i.e. the "419 scam" (label 1).
    \item Utilised for its focus on social engineering tactics to test models' resilience to distinct attacks and psychological language manipulation.
  \end{itemize}
  \item \textbf{Tertiary independent testing dataset: Nazario Spam}:
  \begin{itemize}
    \item The final independent test used, containing purely of phishing emails (label 1).
    \item Although older (2004-2007), its a collation of focused phishing emails to test th emodel on a generalisation of different phishing attacks.
  \end{itemize}
\end{itemize}
