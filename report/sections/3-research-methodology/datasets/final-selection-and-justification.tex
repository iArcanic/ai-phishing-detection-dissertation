% ai-phishing-detection-dissertation/report/sections/3-research-methodology/datasets/final-selection-and-justification.tex

\subsubsection*{Final selection and justification}
From the evaluation above, the selected datasets that were chosen for this research to address the varied nature of phishing attackers are detailed below. A breakdown is provided, that includes their synergies, and close alignment with the identified research gaps.

\begin{itemize}
  \item \textbf{Primary training dataset: Enron}
  \begin{itemize}
    \item Enron's corporate emails provide varied range, in terms of linguistics.
  \end{itemize}
  \item \textbf{Secondary training dataset: SpamAssassin}
  \begin{itemize}
    \item Consists of phishing and spam subsets to serve as a way to test model discrimination.
    \item Helps to reduce the risk of overfitting, as its limited to phishing-only data.
    \item Improves the real-world applicability of the model (\hyperref[research-gap-4]{\uline{\textbf{Research Gap 4}}}).
  \end{itemize}
\item \textbf{Primary testing dataset: Nazario Spam}:
  \begin{itemize}
    \item 
  \end{itemize}
  \item \textbf{Secondary testing dataset: Nigerian Fraud}
  \begin{itemize}
    \item Tests the models resilience on psychological manipulation and social engineering tactics, like phrases of urgency.
    \item Can serve as an adversarial subset, with 100 or so samples being injected into the validation data.
    \item Allows the evaluation of fidelity, e.g. if XAI's explanations pick up on the phrases of urgency.
    \item Strengths user-centric XAI explanations (\hyperref[research-gap-2]{\uline{\textbf{Research Gap 2}}}).
  \end{itemize}
\end{itemize}
