% ai-phishing-detection-dissertation/report/sections/3-research-methodology/evaluation-framework/introduction.tex

The following section presents a rigorous assessment methodology to evaluate the XAI phishing detection models. It validates both the technical performance and operational usability, with respect to human factors, with a multi-faceted evaluation framework, addressing \uline{\textbf{Research gaps} \hyperref[research-gap-1]{\uline{\textbf{1}}}, \hyperref[research-gap-2]{\textbf{2}}, and \hyperref[research-gap-3]{\textbf{3}}}. It builds on evaluation principles from \cite{reddy2023explainable} and \cite{van2024applicability}, focussing on quantative performance metrics, model behaviour on diverse datasets, and a qualitative analysis of generated explanations. The evaluation framework is designed to be adaptable, allowing for the integration of new metrics and methodologies as the field of XAI evolves. This ensures that the framework remains relevant and effective in assessing the performance and usability of XAI models in phishing detection.
