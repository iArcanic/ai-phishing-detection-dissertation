% ai-phishing-detection-dissertation/report/sections/3-research-methodology/evaluation-framework/model-performance-metrics.tex

\subsubsection*{Model performance metrics}
First, this evaluation framework employs standard classification metrics, as well as phishing-specific criteria to serve as an assessment for assessing the detection capabilities of the model. It utilises benchmarking methods from \cite{kapoor2024comparative} and \cite{zamir2020phishing} to evaluate the model's performance.\newline

\noindent Generally, all classification models are evaluated using some baseline metrics, but for this specific use-case, the algorithms are evaluated using a set of standard metrics, which are then adapted to the specific needs of phishing detection.

\begin{itemize}
  \item \textbf{Precision-recall trade-off}:
  \begin{itemize}
    \item \textit{Precision}: $\frac{TP}{TP + FP}$, helps minimise false alarms.
    \item \textit{Recall}: $\frac{TP}{TP + FN}$, prioritises true positives, i.e. catching attacks.
    \item \textit{F2-score}: $\frac{5 \cdot \text{Precision} \cdot \text{Recall}}{4 \cdot \text{Precision} + \text{Recall}}$, emphasises recall over precision.
  \end{itemize}
  \item \textbf{AUC-ROC}:
  \begin{itemize}
    \item Is a measure of the separability of phishing email classes against legitimate email classes.
    \item A critical metric for evaluating the model's performance, given inherent imbalance in datasets \citep{ahmad2024across}.
  \end{itemize}
\end{itemize}

\noindent Phishing-specific metrics are also used to evaluate the model's performance, as they are more relevant to the specific use-case of phishing detection.

\begin{itemize}
  \item \textbf{Cost-sensitive accuracy}:
  \begin{itemize}
    \item Use weighted error metrics from \cite{atlam2022business} to account for the cost of misclassifying phishing emails as legitimate:
    \begin{equation}
      \text{Cost-sensitive accuracy (CSA)} = 1 - \left(\frac{0.2 \cdot FP + 0.8 \cdot FN}{N}\right)
    \end{equation}
    \item Reflects a 4:1 ratio of legitimate to phishing emails, where the cost of misclassifying a phishing email (false positive) is 4 times that of a legitimate email (false negative).
  \end{itemize}
  \item \textbf{Time-to-Detection (TTD)}:
  \begin{itemize}
    \item Measures the latency between email arrival and detection.
    \item Important for real-time detection, especially in high-volume environments, where the target is under 2 minutes \citep{shirazi2022towards}.
  \end{itemize}
  \item \textbf{Attack-type breakdown}:
  \begin{itemize}
    \item Use separate metrics for different attack types, such as:
    \begin{itemize}
      \item URL phishing (precision focus).
      \item Social engineering (recall focus).
      \item Business email compromise (F2-score focus).
    \end{itemize}
  \end{itemize}
\end{itemize}
