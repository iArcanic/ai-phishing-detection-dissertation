% ai-phishing-detection-dissertation/report/sections/3-research-methodology/evaluation-framework/adversarial-testing.tex

\subsubsection*{Adversarial testing}
The system's resilience should be measured against sophisticated phishing tactics, following the framework from \citep{kapoor2024comparative}. Metrics are inspired and adapted from MITRE ATT\&CK\textsuperscript{\textsuperscript{\textregistered}} T1566 (Phishing) \citep{mitre2020phishing}.\newline

\noindent The attack simulation methodology includes the folliwing tactics to test the adversarial robustness of the system.

\begin{itemize}
  \item \textbf{Text-based attacks}:
  \begin{itemize}
    \item Synonym substitution, such as "\textit{urgent}" $\rightarrow$ "\textit{immediate}" \citep{andriu2023adaptive}.
    \item Homoglpyh replacements, like "\textit{Paypal}" $\rightarrow$ "\textit{paypal}", including Unicode attack vectors \citep{greco2023explaining}.
    \item Use generative transformers, like GPT-3.5 for paraphrasing, to aid with semantic preturbations.
  \end{itemize}
  \item \textbf{URL obfuscation}:
  \begin{itemize}
    \item Test subdomain spoofing, like "\textit{login.bank.com}" $\rightarrow$ "\textit{bank.com.login}".
    \item Spoofing with legitimate certificates via HTTPS stripping.
    \item URL shortening with more than 3 redirections.
  \end{itemize}
  \item \textbf{Real-world adversaries}:
  \begin{itemize}
    \item Use the 419 scam templates from the Nigerian Fraud Database \citep{champa2024phishing}.
  \end{itemize}
\end{itemize}

\noindent The defensive capabilities of the system should be evaluated against the following metrics.

\begin{itemize}
  \item \textbf{Detection robustness}:
  \begin{equation}
    R_d = \frac{\text{Detected attacks}}{\text{Total attacks}} \times 100\%
  \end{equation}
  Aiming to reach a target of $R_d \geq 95\%$ for known attack types \citep{atlam2022business}.
  \item \textbf{Explanation stability}:
  \begin{itemize}
    \item Values for SHAP explanatins for different perturbations should be similar ($\Delta SHAP M 5\%$).
    \item The top feature for LIME explanatins should remain consistent across different perturbations.
  \end{itemize}
  \item \textbf{Adaptive threshold performance}:
  \begin{itemize}
    \item FNR should be less than 5\% for known attack types.
    \item Alert fatigue metrics should be less than 5\% for unknown attack types.
  \end{itemize}
\end{itemize}
