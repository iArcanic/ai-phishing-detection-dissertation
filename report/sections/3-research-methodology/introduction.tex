% ai-phishing-detection-dissertation/report/sections/3-research-methodology/introduction.tex

\subsection*{Introduction}
This section concerns itself with the details of the practical implementation concerning this research. A systematic, multi-phase research methodology is adopted, to develop a well evaluated and tested XAI-powered AI phishing detection model. This approach is also designed to address the research gaps identifies as part of the review in \hyperref[sec:2-literature-review]{\uline{\textbf{Section 2}}}, as well as meet the four objectives outlined in \hyperref[sec:1-introduction]{\uline{\textbf{Section 1}}}.\newline

\noindent The methodology first consideres potential phishing email datasets and selects multiple suitable candidates that will be the most helpful in addressing the research question of this project. This data will then need to be preprocessed and cleaned, so a pipeline for this functionality will be defined. We then explore the development of the actual model, selecting machine learning algorithms that have already been proved to perform well in the context of phishing detection. XAI integration into these models will be analysed too. Additionally, an evaluation framework to test the model against will be described, as a way to measure the overall model performance, efficiency, and viability of XAI explanations. Furthermore, all tools required for this implementation will explicitly be listed for future reproducibility purposes.

