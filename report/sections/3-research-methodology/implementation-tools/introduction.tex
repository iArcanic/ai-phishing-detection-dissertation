% ai-phishing-detection-dissertation/report/sections/3-research-methodology/implementation-tools/introduction.tex

This section documents the tools required for all the practical implementation steps detailed previously. The tools used allow for the XAI phishing detection system to be developed systematically and efficiency, ensures reproducibility, and addresses the security and computational limitations identified as part of \hyperref[research-gap-3]{\uline{\textbf{Research Gap 3}}}. The careful selection of the following tools keep in mind.\newline

\noindent The project was implemented in Python 3.x, leveraging the Google Colaboratory environment to interact with and develop the required Jupyter Notebooks as well as access the necessary cloud computational resources, such as GPUs for deep learning model training. Additionally, a range of Python's open source libraries were employed at each stage of the implementation, ranging from data preprocessing to model development and evaluation. The use of these libraries ensures that the implementation is efficient, modular, and utilises existing solutions to common problems.\newline

\noindent This toolchain has been checked against the system requirements that originate from earlier sections where data preprocessing and model development were defined. Version control, as well as cross-compatible frameworks are used to allow for consistency across different deployment platforms -- and for future development or extensibility.
