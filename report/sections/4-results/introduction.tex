% ai-phishing-detection-dissertation/report/sections/4-results/introduction.tex

This section presents the results of the empirical findings from the implementation and evaluation of the two developed phishing detection models as part of the conducted research. The main objective to to report on the performance outcomes of the Random Forest classifier and the fine-tuned DistilBERT model. The insights generated by the XAI techniques, SHAP and LIME, will be showcased too.\newline

\noindent The chapter first displays the performance of the Random Forest model in detail, including its metrics on the internal test set and ability to generalise on independent, external sets (SpamAssassin, Nigerian Fraud, and Nazario). Afterwards, the XAI elements of Random Forest is explored through the generated SHAP explanations, with a presentation of global and local feature importance, and instance-level explanations, with waterfall and force plots. Subsequently, the performance of the fine-tuned DistilBERT model is shown, following the same evaluation framework as the Random Forest model, with results on the internal test set and external sets. Similarly, the DistilBERT model's XAI component is also illustrated, with LIME providing local, word-level explanations for the model's predictions, highlighting words that contributed to the model's classification. Finally, there is a summary on the qualitative feedback performed on the clarity and perceived usefulness of the generated XAI explanations.\newline

\noindent The aforementioned performance metrics, such as accuracy, recall, F1-score (for the phishing class), and ROC AUC scores, along with clasification reports and confusion matrices, will also be presented in this section, where appropriate. XAI results are complemented by a visualisation of the generated explanations to help with breaking down the understanding of model classifications. Following this factual and comprehensive outcome, the next section discusses and analyses the results.
