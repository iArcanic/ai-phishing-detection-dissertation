% ai-phishing-detection-dissertation/report/sections/1-introduction/structure.tex

\noindent The paper is structured as follows: \hyperref[sec:1-introduction]{\uline{\textbf{Section 1}}} outlines this study's objectives and research questions. \hyperref[sec:2-literature-review]{\uline{\textbf{Section 2}}} is the result of a comprehensive literature review into the space of AI-based phishing detection, which neatly progresses onto the relevance of XAI in this field. \hyperref[sec:3-research-methodology]{\uline{\textbf{Section 3}}} discusses the research methodology, and the practical implementation of the XAI-enhanced phishing detection model. \hyperref[sec:4-results]{\uline{\textbf{Section 4}}}. \hyperref[sec:5-discussion]{\uline{\textbf{Section 5}}}. \hyperref[sec:6-conclusion]{\uline{\textbf{Section 6}}}.
