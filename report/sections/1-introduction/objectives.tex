% ai-phishing-detection-dissertation/report/sections/1-introduction/objectives.tex

\subsection{Objectives}

\begin{enumerate}
  \item \uline{\textbf{OBJECTIVE 1}}: "\textit{Conduct a comprehensive review of existing AI phishing detection models}.\label{objective-1}"
  \begin{itemize}
    \item Review existing AI phishing detection systems.
    \item Identify research gaps to address in relation to interpretability in phishing detection.
    \item Explore XAI techniques, like SHAP and LIME, and their applicability for phishing detection
    \end{itemize}
  \item \uline{\textbf{OBJECTIVE 2}}: "\textit{Identify and implement suitable XAI techniques (e.g., SHAP, LIME)}.\label{objective-2}"
  \begin{itemize}
    \item Develop upon a traditional model already being used for phishing detection, e.g. Random Forest or Transformer-based models.
    \item Integrate XAI techniques into these models that can explain the model predictions.
    \item Achieve a competitive accuracy that is either on par or better than existing AI-based phishing detection models.
  \end{itemize}
\item \uline{\textbf{OBJECTIVE 3}}: "\textit{Evaluate the system's performance in terms of accuracy and interpretability}.\label{objective-3}"
  \begin{itemize}
    \item Assess the system on performance metrics such as precision, accuracy, and recall.
    \item Analyse how useful explanations are using standard interpretability metrics.
  \end{itemize}
\item \uline{\textbf{OBJECTIVE 4}}: "\textit{Compare the usability of the XAI phishing detection model with traditional black-boxed models}.\label{objective-4}"
  \begin{itemize}
    \item Conduct small-scale user studies or surveys to determine how effective XAI influences usability and trust -- compared to black-boxed phishing detection models.
    \item Discuss whether or not its worth compromising on accuracy to achieve trust and usability.
  \end{itemize}
\end{enumerate}
